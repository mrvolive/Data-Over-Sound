\chapter{Methodology}

The sound containing the data in a text file is created with the DosSend program. It will be read back with DosRead to see how efficiently it can process the signal without altering the data it contains.

\section{DosSend and DosRead usage}

\subsection{Using DosSend to create a wav file}

The program will convert the input text into an audio signal and save it as \texttt{DosOok\_message.wav}. It will also print characteristics of the signal to the console and display a graphical representation of the signal waveform. It is a sinusoïdal wave modulated using a mix of OOK modulation, quick fade-in transition for 0 to 1 transitions and slow fade-out transition from 1 to 0. This was done to better understand the difference between the two filters' algorithms.

\subsection{Using DosRead}

The program will read, process (through a low pass filter), and analyze audio data from the \texttt{WAV} file. It will then output information about the file and the text corresponding to the binary sequence to a terminal. It will show a graphical representations of the original signal and it's filtered counterpart.

\section{Explanation of the Low-Pass Filters}

\subsection{LPFilter1 : Moving Average Filter}

\subsubsection{Theory behind the Simple Moving Average Filter}

The moving average filter is a basic type of Low Pass FIR (finite-impulse response) filter that is widely employed to smooth out a sequence of data points or a signal. It operates by averaging a set number, n, of input samples at once, and then outputs a single averaged data point.

\subsubsection{Java implementation of the SMA Filter}

\begin{itemize}
	\item The method initializes an array called \texttt{filteredAudio} to store the filtered signal. This array is the same length as the input signal.
	\item The actual filtering is done by averaging a number of samples around each sample in the input signal. 
	\item For each sample in the input signal, the method calculates the sum of the samples in a window centered around the current sample. The size of this window is determined by \texttt{n} and represents the cut off frequency of the filter.
	\item The average is calculated by dividing the sum by the count of samples included in the sum.
	\item This average becomes the new filtered value for the current sample position in the \texttt{filteredAudio} array.
\end{itemize}

\subsection{LPFilter2 : Exponential Moving Average Filter}

\subsubsection{Theory behind the Exponential Moving Average Filter}

The exponential moving average (EMA) filter is a Low Pass, infinite-impulse response (IIR) filter. It prioritizes recent data by giving it more weight and discounting older data in an exponential manner. In contrast to a simple moving average (SMA) this ensures that the trend is maintained by still considering a significant portion of the reactive nature of recent data points.

\subsubsection{Java implementation of the EMA Filter}

\begin{itemize}
	\item The calculates the time constant \texttt{rc} using the formula \texttt{1 / (cutoffFreq * 2 * Math.PI)}. This time constant is used in RC (resistor-capacitor) circuits to determine the filter's response time to changes in the input signal.
	\item The time step \texttt{dt} is calculated as the inverse of the sampling frequency \texttt{1 / sampleFreq}.
	\item The \texttt{alpha} value, which determines the weight given to new samples in the EMA, is calculated using \texttt{dt / (rc + dt)}.
	\item The output signal array, \texttt{outputSignal}, is initialized to have the same length as the input signal.
	\item The first element of the output signal is set to be equal to the first element of the input signal, serving as the initial condition for the EMA.
	\item The method then iterates over each element in the input signal array starting from the second element.
	\item For each element, it applies the EMA filter using the formula : \texttt{outputSignal[i] = outputSignal[i - 1] + alpha * (inputSignal[i] - outputSignal[i - 1])}. This equation takes the previous output value and moves it towards the current input value by a fraction \texttt{alpha}. \texttt{alpha} determines how quickly the filter responds to changes in the input signal. A smaller \texttt{alpha} would make the filter respond more slowly, emphasizing lower frequencies and attenuating higher frequencies more strongly.
	\item After processing all samples, the method returns the outputSignal array, which contains the low-pass filtered signal.
\end{itemize}

\section{Method to monitor the speed of the filters}

Each filter will process 3 files : A short one (two words), A medium one (one paragraph, 97 words), A long one (10 paragraphs, 939 words). Each of the wav files will have been created with DosSend.java and processed within DosRead.java. This will create ideals conditions for the filters as the noise will be minimal. For each files 3 cut off frequencies will be used against a 1kHz signal and each of them must not alter DosRead decoding capability.