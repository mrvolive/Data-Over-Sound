\chapter{Discussion}

Both filters worked correctly for cut off values around 20 Hz making the delta between maximum and minimum amplitude for a period unoticable, going higher however increased that delta as seen by the signal oscillation being visible again, going above 1000Hz made the signal unreadable as it's amplitude was too fluctuent and needed a lower threshold that made noise be read as 1 instead of 0.

As both filters return the same kind of signal after processing a sinusoïd, it can be said that they are both usable for the same kind of work. It is important to note that the EMA filter offers a sharper frequency response rate as it drops and rise quicker than the SMA filter for a low cut off frequency. For a higher cut off frequency value, both filter seems to behave similarly keeping the shape of the original signal but with significant noise.

Performance wise the SMA filter is way behind the EMA Filter with processing getting longer as the number of samples it needs to filter the signal rises. On the other hand, the EMA filter is not impacted by the value for it's cut off frequency making it better than the SMA filter as the frequency of signal to process increases. This is reflected on the time it takes for a filter to process an increasingly bigger file. There is already an 11 ms difference for a file containing 13 characters but it goes up to a 2 seconds difference for a 6307 characters long file. An other way to interpret this data is to consider that it takes the SMA filter 4.27 ms of processing per character while the EMA filter hovers around 0.15 ms per character.