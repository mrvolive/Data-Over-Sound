\chapter{Introduction}

The objective is to examine and compare two low-pass filter designs with the end goal of enhancing the clarity of text data transmitted via audio signals. This comparison is a component of a larger project, the DosOok challenge, which involves the development of the two programs DosSend and DosRead for sending and receiving text data through sound.

In this context, a low-pass filter is crucial as it allows the desired signal to pass while attenuating frequencies that are not needed. The necessity for this comparison comes from the fact that we need our data to be kept intact while making the processing as seemless as possible.

For this study, two distinct low-pass filtering techniques have been selected. The criteria for their evaluation are the speed of processing and the accuracy (the filter will get a pass if the initial text isn't altered after the signal has been processed).

This report will detail the theoretical concepts behind each filter, the process of their implementation in Java, and the outcomes of their performance tests. The aim is to determine which filter provides a superior solution for the problem at hand.